\documentclass[12pt,a4paper]{article}
\usepackage{amsmath, amssymb}
\usepackage{braket}

\usepackage{fancyhdr}
\headheight 18pt
\pagestyle{fancy}
\lhead{Contact and friction}
\chead{}
\rhead{\thepage}
%\lfoot{\footnotesize DNV Software}
%\cfoot{\footnotesize \today}
%\rfoot{\footnotesize MSD TR 2004--XX}

% Equation numbering
\numberwithin{equation}{section}
\numberwithin{table}{section}
\numberwithin{figure}{section}

\newcommand{\R}{\ensuremath{\mathbb{R}}}
\newcommand{\Rd}{\ensuremath{\R^d}}
\newcommand{\Rm}{\ensuremath{\R^m}}
\newcommand{\W}{\ensuremath{\mathcal{W}}}
%\newcommand{\V}{{\mathcal{V}}}
\newcommand{\J}{\ensuremath{\mathcal{J}}}
\newcommand{\Jh}{\ensuremath{\J_h}}

\newcommand{\half}{\ensuremath{\frac{1}{2}}}

\newcommand{\dOmega}{{\partial\Omega}}
\newcommand{\OT}{\Omega\times[0,\tau]}

\newcommand{\pd}[1]{\ensuremath{\partial_{#1}}}
\newcommand{\dt}{\pd{t}}
\newcommand{\di}{\pd{i}}
\renewcommand{\dj}{\pd{j}}
\newcommand{\dl}{\pd{l}}
\newcommand{\dk}{\pd{k}}
\newcommand{\dij}{\pd{ij}}
\newcommand{\dkk}{\pd{kk}}
\newcommand{\pdx}{\pd{1}}
\newcommand{\pdy}{\pd{2}}
\newcommand{\pdz}{\pd{3}}
\newcommand{\pdxx}{\pd{11}}
\newcommand{\pdyy}{\pd{22}}
\newcommand{\pdzz}{\pd{33}}
\newcommand{\pdxy}{\pd{12}}
\newcommand{\pdyz}{\pd{23}}
\newcommand{\pdxz}{\pd{13}}
\newcommand{\D}{\partial}
\newcommand{\dn}{\partial_n}
\newcommand{\grad}{\nabla}
\renewcommand{\div}{\nabla\!\cdot\!}
\newcommand{\tin}{\ \text{in}\ }
\newcommand{\ton}{\ \text{on}\ }
% Function spaces
\newcommand{\Lto}{L^2(\Omega)}
\newcommand{\Ltog}{L^2(\Gamma)}
\newcommand{\Lton}{L^2_0(\Omega)}
\newcommand{\Hen}{H^1}
\newcommand{\HenO}{H^1(\Omega)}
\newcommand{\HenOne}{H^1(\Omega_1)}
\newcommand{\HenTwo}{H^1(\Omega_2)}
\newcommand{\Henn}{H^1_0(\Omega)}
\newcommand{\Henh}{H^1(\Omega_h)}
\newcommand{\HenOh}{H^1(\Omega_h)}
\newcommand{\Hennh}{H^1_0(\Omega_h)}
% Discrete function spaces
\newcommand{\V}{\ensuremath{\mathcal{V}}}
\newcommand{\K}{\ensuremath{\mathcal{K}}}
\newcommand{\Dcomp}{\ensuremath{\mathcal{D}}}
\newcommand{\vPhi}{\ensuremath{\vPhi}}
\newcommand{\Kh}{\ensuremath{\K_h}}
\newcommand{\Vh}{{\mathcal V}_h}
\newcommand{\Wh}{{\mathcal W}_h}

\renewcommand{\L}{{\mathcal L}}
\newcommand{\E}{{\mathcal E}}
\newcommand{\N}{{\mathcal N}}
\newcommand{\Th}{\ensuremath{{\mathcal T}_h}}
\newcommand{\Eh}{{\mathcal E}_h}
\newcommand{\Ehi}{{\mathcal E}_{h,i}}
\newcommand{\Ehone}{{\mathcal E}_{h,1}}
\newcommand{\Ehtwo}{{\mathcal E}_{h,2}}
\newcommand{\Ehthree}{{\mathcal E}_{h,3}}
\newcommand{\Nh}{{\mathcal N}_h}
\newcommand{\F}{\ensuremath{{\mathcal F}}}
\newcommand{\G}{\ensuremath{{\mathcal G}}}

\newcommand{\Lagrange}{\ensuremath{\mathcal L}}

\newcommand{\Ttilde}{\widetilde{T}}
\newcommand{\dT}{\partial T}

\newcommand{\intO}{\int_\Omega\!\!}
\newcommand{\intdO}{\int_{\partial\Omega}\!\!}
\newcommand{\intT}{\int_T\!\!}
\newcommand{\intG}[1][0]{\int_{\Gamma_{#1}}\!\!}
\newcommand{\intGc}{\intG[c]}
\newcommand{\intdT}{\int_{\partial T}\!\!}
\newcommand{\intE}{\int_E}
\newcommand{\intt}{\int_{-t/2}^{t/2}}
\newcommand{\intb}{\int_{-b/2}^{b/2}}
\newcommand{\sumTinTh}{\sum_{T\in\Th}}
\newcommand{\sumEinEh}{\sum_{E\in\Eh}}
\newcommand{\sumEinEhi}{\sum_{E\in\Ehi}}
\newcommand{\sumEinEhone}{\sum_{E\in\Ehone}}
\newcommand{\sumjm}{\sum_{j=1}^m}

\renewcommand{\epsilon}{\varepsilon}
\renewcommand{\phi}{\varphi}

\newcommand{\strain}[1][]{\ensuremath{\epsilon_{#1}}}
\newcommand{\epsij}{\strain[ij]}
\newcommand{\epskk}{\strain[kk]}
\newcommand{\epskl}{\strain[kl]}
\newcommand{\epsji}{\strain[ji]}

\newcommand{\stress}[1][]{\ensuremath{\sigma_{#1}}}
\newcommand{\sigij}{\stress[ij]}
\newcommand{\sigkl}{\stress[kl]}
\newcommand{\sigji}{\stress[ji]}
\newcommand{\djstress}[1][]{\ensuremath{\sigma_{#1,j}}}
\newcommand{\djsigij}{\djstress[ij]}

\newcommand{\momentum}[1][]{\ensuremath{M_{#1}}}
\newcommand{\momij}{\momentum[ij]}

\providecommand{\abs}[1]{\lvert #1 \rvert}
\providecommand{\babs}[1]{\bigl\lvert #1 \bigr\rvert}
\providecommand{\norm}[1]{\lVert #1 \rVert}
%\newcommand{\norm}[1]{\| #1\|}
\providecommand{\dualp}[2]{\langle #1, #2 \rangle}

\newcommand{\infvinV}{\ensuremath{\inf_{v\in \V}}}
\newcommand{\supvinV}{\ensuremath{\sup_{v\in \V}}}
\newcommand{\supwinW}{\ensuremath{\sup_{w\in \W}}}
\newcommand{\supvsinVs}{\ensuremath{\sup_{v^*\in \V^*}}}
\newcommand{\supwsinWs}{\ensuremath{\sup_{w^*\in \W^*}}}

\newcommand{\Kinv}{\ensuremath{K^{-1}}}

\newcommand{\dx}{{\,dx}}
\newcommand{\ds}{{\,ds}}

\newcommand{\uh}{\ensuremath{U}}
\newcommand{\uph}{\ensuremath{U^*}}
\newcommand{\ulambda}{\ensuremath{u(\lambda)}}
\newcommand{\ulambdah}{\ensuremath{u(\lambda+h)}}
\newcommand{\ulh}{\ensuremath{u(h)}}
\renewcommand{\forall}{\text{for all }}
\newcommand{\qforall}{\quad\text{for all }}

\title{Numerical solution of a contact problem with {Coulomb} friction}

\author{Torgeir Rusten}

\date{\today}

\begin{document}

\maketitle

\section{Introduction}
\label{sec:introduction}

The present report discuss the numerical solution of contact problems
with Coulomb friction. The small displacement assumption is used, i.e.\
the displacements are sufficiently small to neglect the differences in
the body before and after deformation.  First the formulation of a
contact problem with friction is recalled, then a finite element
methods is introduced.  Finally som methods for the solution of the
resulting algebraic problem using methods from mathematical
programming is discussed.  Contact and friction problems in elasticity
are discussed in~\cite{kikuchi88:contac-elast}.


\section{Linearized elasticity}

The body of interest occupies the domain $\Omega\subset\mathbb{R}^d$ with boundary $\dOmega$ and outward pointing unit normal vector $\nu = (\nu_1, \ldots, \nu_d)$. In practice the
dimension $d$ is two or three.  The \emph{displacement} of the body is denoted
$u=(u_1,\ldots,u_d)$ and the components of the \emph{small strain tensor} is defined by
\begin{equation}
\epsij(u) = \frac{1}{2} (u_{i,j} + u_{j,i}),
\end{equation}
for $i,j=1\ldots,d$.  Furthermore, the symmetric \emph{Cauchy stress tensor} is denoted
\sigij, for $i,j=1\ldots,d$.

We assume that the material is linear and homogeneous
\begin{equation}\label{hook}
\sigij(u) = C_{ijkl}\epskl(u)
\end{equation}
for $i,j=1\ldots,d$. The tensor $C_{ijkl}$ posess the follows symmetry conditions
\[ C_{ijkl} = C_{jikl}\quad C_{ijkl}=C_{ijlk}\quad\text{and}\quad
C_{ijkl} = C_{klij}.\]
Furthermore, there exist a positive constant $c$ such that
\begin{equation}
\label{eq:HookPositive}
C_{ijkl}\tau_{ij}\tau_{kl} \ge c \tau_{ij}\tau_{kl}
\end{equation}
for all symmetric tensors $\tau_{ij}$ and for almost all $x$ in $\Omega$.

For an isotropic material only two of the 21 parameters are independent constants.
Using the Lam\'{e} coefficients $\lambda$ and $\mu$ we have
\begin{equation}
  \label{eq:isotropStressStrain}
  \sigij(u) = 2\mu\, \epsij(u) 
  + \lambda\, \delta_{ij}\, \epskk(u).
\end{equation}

On the boundary $\dOmega$ the normal stress $\stress^\nu$ is given by
\[ \stress^\nu = \sigij \nu_i \nu_j \]
and the tangential stress $\stress^\tau$ is given by
\[ \stress[i]^\tau = \sigij \nu_j - \stress^\nu \nu_i, \]
for $i=1,\ldots,d$.  In a similar way, let the normal displacement be
\[ u^\nu = u_i \nu_i \]
and the tangential displacement be
\[ u^\tau_i = u_i - u^\nu \nu_i, \]
for $i=1,\ldots,d$.

The boundary $\dOmega$ consist of two disjoint parts $\Gamma_0$ and $\Gamma_1$.  On $\Gamma_0$ the displacement vanish and on $\Gamma_1$ the stress vector is $g$ is prescribed.

In order to give a precise mathematical formulation the space
\begin{equation}
\V = \{ v\in (\HenO)^d \colon v = 0 \ton\Gamma_0\}.
\end{equation}
is used.  On $(\HenO)^d$ introduce the bilinear form
\begin{equation}
\label{eq:Aelasticity}
A(u,\phi) = \intO C_{ijkl}\epskl(u)\epsij(\phi)\dx,
\end{equation}
and the linear functional
\begin{align}
\label{eq:Felasticity}
F(\phi) = & \intO f_i\phi_i\dx
        + \intG[1] g_i \phi_i\ds. \notag
\end{align}

The functional
\begin{equation}
        J(\phi) = \frac{1}{2}A(\phi,\phi) - F(\phi).
\end{equation}
represent the total \emph{potential energy} of the body subject to the
displacement $\phi$.  The problem of linearized elasticity is to find
the displacement $u$ minimizing the potential energy functional $J$,
i.e.,
\begin{equation}
  \label{eq:UnconstrainedMin}
  \min_{\phi\in\V} J(\phi).
\end{equation}
It is well known that this has a unique solution. Furthermore, the unique minimizer $u\in\V$ satisfies
\begin{equation}
  A(u, \phi) = F(\phi)\quad\text{for all $\phi$ in $\V$}
\end{equation}


\section{Constriant minimization}

In unilateral contact the displacement of the body is restricted by a rigid boundary, e.g.\ a elastic body is resting on a rigid foundation, i.e.\ the elastic body can not penetrate into the rigid foundation. In general multibody contact one elastic body can not penetrate into another elastic body. This impose a kinematic constraint on the displacement. Below we show that this amount to introduce a closed convex subspace $\K$\ of $\V$\ and minimize the potenetial energy over $\K$:
\begin{equation}
  \label{ineq:potential_convex}
  \min_{\phi\in\K} \frac{1}{2}A(\phi,\phi) - F(\phi)
\end{equation}
It is shown below that for contact analysis the convex space $\K$ is also a cone.

It can be shown that the above constratint minimization problem has a unique solution, see e.g.\ Glowinski \cite{Glowinski1984:NonlinearVariational}. Moreover, the solution $u \in K$ also satisfies the variational inequality
\begin{equation}
  \label{ineq:varIneq}
  A(u,\phi-u) \ge F(\phi-u) \quad\text{for all $\phi$ in $\K$.}
\end{equation}

Since \K\ is a cone, $\phi = 0$ and $\phi = 2 u$ is in \K\ and it follows that
\begin{equation}
  \label{eq:varIneqSol}
  A(u,u) = F(u)
\end{equation}
and
\begin{equation}
  \label{ineq:varIneqSol}
  A(u,\phi) \ge F(\phi) \quad\text{for all $\phi$ in \K}
\end{equation}
For further discussion cf.\ Glowinski 1984 \cite{Glowinski1984:NonlinearVariational}, Lemma 4.1.

\section{Unilateral contact}

In unilateral contact the displacement of the body is restricted by a rigid boundary, e.g.\ a elastic body is resting on a rigid foundation.  In this case the boundary of the domain $\Omega$ consists of three disjoint parts, $\Gamma_0$, $\Gamma_1$ and $\Gamma_c$. Here $\Gamma_3$ represent the part that may be in contact with the rigid foundation. In particular the the normal component of the displacement must be less or equal to zero, otherwise the body will penetrate the rigid foundation, so
\begin{equation}
  u^\nu \le 0 \quad\text{on $\Gamma_c$}
\end{equation}

Introduce
\begin{equation}
  \label{eq:contactSubspace}
  \K = \{ \phi\in \V \colon \phi^\nu \le 0 \text{ a.e.\ on $\Gamma_c$}\}
\end{equation}
Clearly \K\ is a convex subspace of $\V$, and also a cone with vertex zero.
We consider linearized elasticity together with the convex subspace \eqref{eq:contactSubspace}, i.e.\ the constriant minimization problem \eqref{ineq:potential_convex}

Now, the inequality \eqref{ineq:varIneqSol} and equality \eqref{eq:varIneqSol} is used to characterize the solution of the contact problem.

Let $\V_0$ be the subset of $\V$ of functions vanishing normal trace on $\Gamma_c$ in addition to $\Gamma_0$. Clearly $\V_0$ is a subset of $\K$ and also a linear space, i.e.\
\begin{equation}
  A(u,\phi) = F(\phi) \quad\text{for all $\phi\in\V_0$}
\end{equation}

Using Green's theorems
\begin{equation}
  A(u,\phi) = \intO -\djsigij \phi_i \dx
  + \intG[1] \nu_j \sigij \phi_i \ds + \intG[c] \sigij^\tau \phi^\tau_i \ds
\end{equation}
where we have used that the normal component vanish on the contact boundary. Inside the domain $\Omega$ the equilibrium condition holds
\begin{equation}
  -\djsigij = f \quad\text{in $\Omega$}
\end{equation}
and on $\Gamma_1$ the stress vector $\nu_j \sigij$ equals $g_i$. Furthermore, on the contact boundary the tangential stress $\sigij^\tau$ vanish.  Consequently
\begin{equation}
  \intG[c] \sigma^\nu \phi^\nu \ds \ge 0 \quad\text{forall $\phi$ such that $\phi^\nu \le 0$}
\end{equation}
and it is clear that the normal stress on the contact boundary must be less or equal to zero. Furthermore, for the solution
\begin{equation}
  \intG[c] \sigma^\nu u^\nu \ds = 0
\end{equation}
hence on a point on $\Gamma_c$ either $\sigma^\nu = 0$ or $u^\nu = 0$.

To sum up, on the contact boundary either the normal displacement is zero or the normal stress is zero. At a point on the contact boundary where the normal displacement is zero the body is in contact with the rigid foundation. At a point where the normal displacement is negative it is not contact and the boundar is free.  Inside $\Omega$ the elastic body is in equilibrium and the on $\Gamma_1$ the stress is equal to the given traction.


\subsection{Fenchel duality}

We refer to the constraint minimization problem \eqref{ineq:potential_convex}with the convex subspace given by \eqref{eq:contactSubspace} as the primal problem. The purpose of this section is to introduce the corresponding dual problem corresponding.

First, introduce the space
\begin{equation}
  \W = \set{\mu \in (L^2(\Omega))^{d,d} | \mu_{ij} = \mu_{ji}
    \text{ for } i\ne j } 
\end{equation}
Here we can identify $\W^*$, the dual space of $\W$, by $\W$.  Let $\strain \in \L(\V,\W)$
be given by
\begin{equation}
  \epsij(v) = \half (v_{i,j} + v_{j,i})
\end{equation}
The dual operator
$\strain^*$ in $\L(\W,\V^*)$ is defined by
\begin{equation}
  \strain^* \mu(v) = \intO \mu_{ij} \epsij(v) \dx
\end{equation}
%where $\dualp{\cdot}{\cdot}$ is the duality pairing between $\V^*$ and $\V$.  

Define the functional $F$ on \V\ and $G$ on \W\ by
\begin{equation}
  \F(v) =
  \begin{cases}
    -F(v) & \text{if } v \in \K \\
    \infty & \text{else;}
  \end{cases}
\end{equation}
and
\begin{equation}
  \G(\mu) = \half \intO \mu_{ij} C_{ijkl} \mu_{kl} \dx
\end{equation}
Then
\begin{equation}
  \label{ineq:primalFenchel}
  \infvinV \F(v) + \G(\strain(v))
\end{equation}
is equivalent to~(\ref{eq:PrimalVar}).

We introduced the functional
\begin{equation}
\varPhi(v,\mu) = \F(v) + \G(\epsilon(v) + \mu)
\end{equation}
The conjugate functional is
\begin{equation}
  \varPhi^*(v^*, \mu) = \F^*(v^* - \strain^* \mu) + \G^*(\mu)
\end{equation}
where we have used that the dual of $W$ can be identified with $W$ itself. It can be shown that
\begin{equation}
	\label{eq:infsup}
  \inf_{v\in \V} \F(v) + \G(\strain(v))
  = \sup_{\mu\in \W} -\F^*(-\strain^* \mu) - \G^*(\mu)
\end{equation}
The optimal solutions of are denoted $u$ and $ \stress$.


Consider the supremum of the right hand side of equation~\eqref{eq:infsup}, for the optimal solutions $\stress$
\begin{equation}
  \F^*(-\strain^*(\stress)) = \sup_{v \in \V} \bigl(-\strain^* \stress(v)
   - \F(v)\bigr)
  = \sup_{v\in \V} \bigl(-\intO \stress[ij]\epsij(v) \dx - \F(v) \bigr)
\end{equation}
If $v$ is not in $\K$, $-\F(v)$ is minius infinity, consequently $v$ must be in the convex subspace $\K$. If $v\in \K$
\begin{equation}
  \sup_{v\in\K} \bigl(-\intO \stress[ij]\epsij(v) \dx + F(v) \bigr)
\end{equation}
The first term can be written
\begin{equation}
  \intO \stress[ij] \epsij(v) \dx = \intO \stress[ij] v_{i,j} \dx
  = \int_\Gamma \nu_j \stress[ij] v_i \ds - \intO \stress[ij,j] v_i \dx
\end{equation}
Combining this with the definition of $F$, we obtain
\begin{equation}
  \intO \stress[ij,j] v_i \dx - \intG[1] \nu_j \stress[ij] v_i \ds 
  - \intG[c] \stress[i]^\tau v^\tau_i + \stress^\nu v^\nu \ds + \intO f_i v_i \dx
  +   \intG[1] g_i v_i\ds.
\end{equation}
The supremum is infinit, unless
\begin{equation}
  \label{eq:stressEq}
  \begin{split}
    -\stress[ij,j] &= f_i \quad\text{in $\Omega$} \\
    \nu_j \stress[ij] &= g_i \quad\text{on $\Gamma_1$} \\
    \stress[i]^\tau &= 0 \quad\text{on $\Gamma_c$}      
  \end{split}
  \quad\text{for $i=1,\ldots,d$}
\end{equation} 
If, in addtion, $\stress^\nu \le 0$ on $\Gamma_c$
\begin{equation}
  \sup_{v\in\K} \F(-\strain^*(\stress))
  = \sup_{v\in K} \intG[c] \stress^\nu v^\nu \ds = 0
\end{equation}
If the above conditions are not stisfied, $\F(-\strain^*(\mu)) = \infty$.

Furthermore, evaluating the conjugate of $\G$ at the optimal point $\stress$
\begin{equation}
  \G^*(\stress) = \sup_{\mu\in W} \bigl(\intO \stress[ij] \mu_{ij} \dx
   - \half \intO \mu_{ij} C_{ijkl} \mu_{kl} \dx \bigr)
\end{equation}
and the supremum is attained when
\begin{equation}
  C_{ijkl} \mu_{kl} = \stress[ij] \quad\text{in $\Omega$ and for all $i,j$}
\end{equation}
The tensor $C$ is invertible and the inverse is denoted $E$, hence
\begin{equation}
  \G^*(\stress) = \half \intO \stress[ij] E_{ijkl} \stress[kl] \dx
\end{equation}
Using the above, the dual problem is
\begin{equation}
  \sup_{\substack{\mu\in\W \\ \mu^{\nu} \le 0, \text{ on $\Gamma_c$}, \\
   \text{and \eqref{eq:stressEq} holds.}}}
  -\half \intO \mu_{ij} E_{ijkl} \mu_{kl} \dx
\end{equation}

For a given $\lambda$ in $H^{-\half}(\Gamma_c)$ let $\ulambda$ satisfy
\begin{equation}
  A(\ulambda, \phi) = F(\phi) - \dualp{\lambda}{\phi^\nu} 
  \quad\text{for all $\phi$ in $V$}
\end{equation}
Now
\begin{equation}
  \intO \mu_{ij} E_{ijkl} \mu_{kj} \dx
  = \intO  \epsij(\ulambda) C_{ijkl} \epskl(\ulambda) \dx
  = A(\ulambda, \ulambda)
\end{equation}
and the dual problem takes the form
\begin{equation}
  - \inf_{\substack{\lambda \in H^{-1/2}(\Gamma_c) \\ \lambda \ge 0}}
  A(\ulambda, \ulambda)
\end{equation}

\section{TMP}

The extremity condition $-\strain^*(\stress) \in \partial \F(u)$, by def
\begin{equation}
	\F(v) - \F(u) \geq \dualp{-\strain^*(\stress)}{v-u} \quad\text{for all $v\in\V$}
\end{equation}
Clearly $u$ and $v$ must be in $\K$, using definitions
\begin{equation}
	\dualp{\stress}{\strain(v-u)} \geq F(v) - F(u) \quad\text{for all $v\in\K$}
\end{equation}
In addition, $\strain^* \in \partial \G(\strain(u))$, i.e.\
\begin{equation}
  \stress[ij] = C_{ijkl} \strain[kl](u)
\end{equation} 

We call
\begin{equation}
  \infvinV \varPhi(v, 0)
\end{equation}
the primal problem and
\begin{equation}
  \sup_{\mu\in \W} -\varPhi^*(0,\mu)
\end{equation}
is the dual problem with respect to $\varPhi$.  Here
\begin{equation}
  \varPhi^*(v^*,\mu) = \F^*(v^*-\epsilon^*(\mu)) + \G^*(\mu)
\end{equation}
where $\F^*$ and $\G^*$ are the conjugate functionals of $\F$ and $\G$.  Consequently
\begin{equation}
  \sup_{\mu\in \W} -\varPhi^*(0,\mu)
  = \sup_{\mu\in \W} -\F^*(-\epsilon^*(\mu)) - \G^*(\mu)
\end{equation}

It is known that
\begin{equation}
  \label{eq:PrimalDualNoGap}
  \infvinV \F(v) + \G(\strain(v))
  = \sup_{\mu\in \W} \{ - \F^*(-\strain^* \mu) - \G^*(\mu) \}
\end{equation}
The solution to the primal problem is denoted $u$, as above, and the solution of the dual
problem is denoted $\sigma$.  Recall that $u-u_l\in \K$. 


\subsection{Lagrange multipliers}

Since the constraint is only on the contact boundary in make sense to introduce the function $h : \V \mapsto H^{1/2}(\Gamma_c)$ as the normal trace function
\begin{equation}
  h(\phi) = \nu_i\cdot\phi_i = \phi^\nu \quad \text{on $\Gamma_c$}
\end{equation}
and introduce the space
\begin{equation}
  K = \set{\lambda \in H^\half(\Gamma_c) | \lambda \le 0}
\end{equation}
We now consider the minimization problem 
\begin{equation}
  \begin{split}
    &\min_{v\in\V} J(v) \\
    &\text{subject to $h(v) \in K$}
  \end{split}
\end{equation}

It can be shown that there exists a function $u\in\V$ and a Lagrange multiplier
$\lambda^* \in \bigl(H^{\half}(\Gamma_c)\bigr)^*$ such that
\begin{align}
  &\dualp{J^\prime(u) + \lambda^* \circ \nu\cdot}{v} = 0 \quad\text{for all $v\in \V$} \\
  &\dualp{\lambda^*}{\lambda}_{\Gamma_c} \le 0 \quad\text{for all $\lambda\in K$} \\
  &\dualp{\lambda^*}{\nu\cdot u}_{\Gamma_c} = 0
\end{align}
Or, equivalently
\begin{align}
  & A(u,v) - F(v) + \dualp{\lambda^*}{\nu\cdot v}_{\Gamma_c} = 0
  \quad\text{for all $v\in \V$} \\
  &\dualp{\lambda^*}{\nu\cdot v}_{\Gamma_c} \le 0 \quad\text{for all $v\in \K$} \\
  &\dualp{\lambda^*}{\nu\cdot u}_{\Gamma_c} = 0
\end{align}



\section{A mathematical model of a contact problem with Coulomb friction}
\label{sec:math-model-cont}

Let $\Omega$ be a body in \Rd with boundary $\dOmega$ and unit outward normal vector
$\nu$.  The boundary is split into three disjoint parts, $\Gamma_0$, $\Gamma_1$ and
$\Gamma_c$, such that $\dOmega = \overline{\Gamma}_0 \cup \overline{\Gamma}_1 \cup
\overline{\Gamma}_c$.  The normal component of a vector function $v$ at the boundary is
$v^\nu = v\cdot\nu$ and the tangential component is $v^\tau = v - v^\nu \nu$.  The
traction at the boundary is $\sigij\nu_j$, with normal component $\sigma^\nu =
\sigij\nu_j\nu_i$ and tangential component $\sigma_i^\tau = \sigij\nu_j - \sigma^\nu \nu$.
The body is fixed on $\Gamma_0$, the traction is given on $\Gamma_1$ and $\Gamma_c$ is the
friction and contact boundary.

In unilateral contact one assume that the body can not penetrate
through $\Gamma_c$, i.e.\ in the normal direction the displacement can
be negative but not positive, i.e.\
\begin{equation}
  u^\nu \le 0
\end{equation}
In general one can allow for an initial distance to a barrier. Let $l$
be the distance function, i.e.\ for the point $x$ on $\Gamma_c$ the
distance to the barrier is $l(x)$.  In this case the kinematic contact
condition takes the form $u^\nu \le l$.

The unilateral contact conditions are
\begin{equation}
  \label{eq:contact}
  u^\nu \le l \quad\text{and}\quad \sigma^\nu = \sigij(u)\nu_i \nu_j \le 0
  \quad\text{on }\Gamma_c
\end{equation}
Note that at points of $\Gamma$ where the bodies are in contact, $u_\nu=$l, while at
points where there is no contact the normal stress vanish, $\sigma_\nu = 0$.
Consequently
\begin{equation}
  \label{eq:7}
  (u^\nu - l) \;\sigma^\nu = 0 \quad\text{on }\Gamma_c.
\end{equation}

In order to present Coulomb's law of friction, let $\beta>0$ be the coefficient of Coulomb
friction and consider a point $x$ on the boundary $\Gamma_c$.  If
\begin{equation}
  \label{eq:9}
  \abs{\sigma^\tau(x)} < -\beta \sigma^\nu(x) \quad\text{then}\quad
  u^\tau(x) = 0
\end{equation}
and if
\begin{equation}
  \label{eq:10}
  \abs{\sigma^\tau(x)} = -\beta \sigma^\nu(x)
\end{equation}
then there exist $\lambda(x)>0$ such that
\begin{equation}
  u^\tau(x) = -\lambda(x) \sigma^\tau(x)
\end{equation}

The space of \emph{virtual displacements} are
\begin{equation}
  \label{eq:VirtualDisplacementsUni}
  \V = \{v\in(\HenO)^d \colon v = 0 \text{ on }\Gamma_0 \}
\end{equation}
and the set of \emph{kinematically admissible} displacements are
\begin{equation}
  \label{eq:KinematicallyAdmisible}
  \K = \{v\in \V \colon v^\nu \le 0 \quad\text{on }\Gamma_c \}
\end{equation}
The set \K\ is a closed and convex subset of \V. It is also a cone
with vertex at zero.  Let $u_l$ be the extension of $l$ into $\V$ such
that $u_l^\nu = l$ on $\Gamma_c$.  Note that if $v-u_l\in K$ we have
$v^\nu-u_l^\nu \le 0$ on $\Gamma_c$, i.e.\ $v^\nu \le l$ on
$\Gamma_c$.

In general the contact condition can be between two bodies, i.e.\ $\Omega =
\Omega_1\cup\Omega_2$, with a common boundary denoted by $\Gamma_c$, i.e.\ $\Gamma_c =
\bar{\Omega}_1\cap\bar{\Omega}_2$.  We choose the unit normal on $\Gamma_c$ pointing from
$\Omega_1$ to $\Omega_2$ as the unit normal on the common boundary.  Note that in general
it may be an initial gap between the two bodies, in this case it will not be a common
boundary, however whenever the normals have the same direction it makes sense to consider
$\Gamma_c$ on $\Omega_1$ the common boundary.  In this case the space of \emph{virtual
  displacements} is 
\begin{equation}
  \label{eq:VirtualDisplacements}
  \V = \{v\in(\HenOne)^d+(\HenTwo)^d \colon v = 0 \text{ on }\Gamma_0 \}
\end{equation}

In the two body case the contact condition is
\begin{equation}
  u_1^\nu \le u_2^\nu + l \quad \text{on $\Gamma_c$}
\end{equation}
where $l$ is the gap measured along the unit normal vector $\nu$.  In this case the
\emph{kinematically admissible} displacements are
\begin{equation}
  \label{eq:KinematicallyAdmisibleGap}
  \K = \{v\in \V \colon [v]^\nu \le 0 \quad\text{on }\Gamma_c \}
\end{equation}
where $[v]^\nu = (v_1 - v_2)\cdot \nu$ on $\Gamma_c$.  Let $u_l$ be an extension of $l$ into
\V\ such that $[u_l] = l$ on $\Gamma_c$.  Then if $v-u_l \in \K$ we have
$[v^\nu]-[u_l^\nu] \le 0$ on $\Gamma_c$, i.e.\ $v_1^\nu \le v_2^\nu + l$ on $\Gamma_c$.

Recall that $v^\nu = \nu\cdot v$, we use $\nu\cdot$ to represent the normal trace operator
of a function in \V, i.e.\ $\nu\cdot : \V \mapsto H^{\half}(\Gamma_c)$.

The \emph{potential energy} functional is given by
\begin{equation}
  \label{eq:PotentialEnergy}
  J(v) = \half A(v,v) - F(v) + j(v).
\end{equation}
Here $A$ represent the internal strain energy, $F$ the work of applied forces and $j$ the
work of the friction energy.

The bilinear form modelling the internal energy is given by
\begin{equation}
  \label{eq:InternalEnergy}
  A(v,\phi) = \intO \sigij(v) \epsij(\phi)\dx
\end{equation}
Here $\sigma$ is the stress tensor and $\epsilon$ is the symmetric part of the
displacement gradient.  In the sequel we assume that
\begin{equation}
  \label{eq:29}
  \sigij(v) = C_{ijkl}\epskl(v)
\end{equation}
The linear functional $F$ is given by
\begin{equation}
  \label{eq:30}
  F(v) = \intO f_i v_i \dx + \intG[1] t_i v_i \ds
\end{equation}
where $f_i$ are body forces and $t_i$ are given traction forces.

If the normal contact stress $\sigma^\nu\le 0$ is known a priori, the slip bound
$g=-\beta \sigma^\nu$ can be evaluated.  The nondifferentiable function representing the
work of the friction force are then
\begin{equation}
  \label{eq:FrictionWork}
  j(v) = \intGc g \norm{v_\tau} \ds
\end{equation}
where $\norm{\cdot}$ is a vector norm.

The \emph{primal variational formulation} is: Find $u-u_l\in\K$ such that
\begin{equation}
  \label{eq:PrimalVar}
  J(u) \le J(v) \quad\forall v-u_l\in \K.
\end{equation}

Note that for a given $g$ this minimization problem has a unique solution $u=u(g)$.  Thus,
for every $g\in L^\infty(\Gamma_c)$, $g\ge 0$, the normal contact stress is
$-\beta\sigma^\nu(u(g))$.  Note that if the a priori choice of the normal contact stress
was correct,
\begin{equation}
  \label{eq:12}
  g = \Phi(g) = -\beta\sigma^\nu(u(g))
\end{equation}
Consequently the contact problem with Coulomb friction can be solved using a fixed point
iteration with the operator $\Phi(g)$ defined above.  In each fixed point step the contact
problem~(\ref{eq:PrimalVar}) with given friction must be solved.

Note that the formulation of the contact problem is a minimization problem of the
form~(\ref{eq:PrimalVar}) with the potential energy function given by
\begin{equation}
  \label{eq:PotentialEnergyContact}
  J(v) = \half A(v,v) - F(v)
\end{equation}
and $v-u_l\in\K$.  This is equivalent to the variational inequality:
Find $u-u_l\in\K$ such that
\begin{equation}
  \label{eq:PrimalVarIneq}
  A(u,v-u) \ge F(v-u) \qforall v - u_l\in \K
\end{equation}

With $w=u-u_l\in \K$ and $\phi=v-u_l \in \K$ the variational inequality problem becomes
\begin{equation}
  A(w, \phi-w) \ge F(\phi-w) - A(u_l, \phi-w) \quad\text{for all $\phi\in\K$}
\end{equation}

An alternative formulation is
\begin{equation}
  \begin{split}
    &\min_{v\in\V} J(v) \\
    &\text{subject to $g(v) \in K$}
  \end{split}
\end{equation}
where $g(v) = \nu\cdot v - l$ and $K$ the closed convex cone defined by
\begin{equation}
  K = \set{\lambda \in H^\half(\Gamma_c) | \lambda \le 0}
\end{equation}
It can be shown that there exists a function $u\in\V$ and a Lagrange multiplier
$\lambda^* \in \bigl(H^{\half}(\Gamma_c)\bigr)^*$ such that
\begin{align}
  &\dualp{J^\prime(u) + \lambda^* \nu\cdot}{v} = 0 \quad\text{for all $v\in \V$} \\
  &\dualp{\lambda^*}{\lambda}_{\Gamma_c} \le 0 \quad\text{for all $\lambda\in K$} \\
  &\dualp{\lambda^*}{\nu\cdot u - l}_{\Gamma_c} = 0
\end{align}
Or, equivalently
\begin{align}
  & A(u,v) - F(v) + \dualp{\lambda^*}{\nu\cdot v}_{\Gamma_c} = 0
  \quad\text{for all $v\in \V$} \\
  &\dualp{\lambda^*}{\nu\cdot v - l}_{\Gamma_c} \le 0 \quad\text{for all $v-u_l\in \K$} \\
  &\dualp{\lambda^*}{\nu\cdot u - l}_{\Gamma_c} = 0
\end{align}


\section{Fenchel duality}

The purpose of this section is to introduce the dual problem corresponding to \eqref{ineq:potential_convex}.

First, introduce the space
\begin{equation}
  \W = \set{\mu \in (L^2(\Omega))^{d,d} | \mu_{ij} = \mu_{ji}
    \text{ for } i\ne j } 
\end{equation}
Here we can identify $\W^*$, the dual space of $\W$, by $\W$.  Let $\strain \in \L(\V,\W)$
be given by
\begin{equation}
  \epsij(v) = \half (v_{i,j} + v_{j,i})
\end{equation}
The dual operator
$\strain^*$ in $\L(\W,\V^*)$ is defined by
\begin{equation}
  \dualp{\strain^*(\mu)}{v} = \intO \mu_{ij} \epsij(v) \dx
\end{equation}
where $\dualp{\cdot}{\cdot}$ is the duality pairing between $\V^*$ and
$\V$.  

Define the functional $F$ on \V\ and $G$ on \W\ by
\begin{equation}
  \F(v) =
  \begin{cases}
    -F(v) & \text{if } v-u_l \in \K \\
    \infty & \text{else;}
  \end{cases}
\end{equation}
and
\begin{equation}
  \G(\mu) = \half \intO \mu_{ij} C_{ijkl} \mu_{kl} \dx
\end{equation}
Then
\begin{equation}
  \infvinV \F(v) + \G(\strain(v))
\end{equation}
is equivalent to~(\ref{eq:PrimalVar}).

We introduced the functional
\begin{equation}
\varPhi(v,\mu) = \F(v) + \G(\epsilon(v) + \mu)
\end{equation}
We call~(\ref{eq:PrimalVar}) the primal problem and
\begin{equation}
  \sup_{\mu\in \W} -\varPhi^*(0,\mu)
\end{equation}
is called the dual problem with respect to $\varPhi$.  Here
\begin{equation}
  \varPhi^*(v^*,\mu) = \F^*(v^*-\epsilon^*(\mu)) + \G^*(\mu)
\end{equation}
where $\F^*$ and $\G^*$ are the conjugate functionals of $\F$ and $\G$.  Consequently
\begin{equation}
  \sup_{\mu\in \W} -\varPhi^*(0,\mu)
  = \sup_{\mu\in \W} -\F^*(-\epsilon^*(\mu)) - \G^*(\mu)
\end{equation}

It is known that
\begin{equation}
  \label{eq:PrimalDual}
  \infvinV \F(v) + \G(\strain(v))
  = \sup_{\mu\in \W} \{ - \F^*(-\strain^* \mu) - \G^*(\mu) \}
\end{equation}
The solution to the primal problem is denoted $u$, as above, and the solution of the dual
problem is denoted $\sigma$.  Recall that $u-u_l\in \K$. 

By definition
\begin{equation}
  \F^*(-\strain^* \mu) = \supvinV \{\dualp{-\strain^* \mu}{v} - \F(v) \}
\end{equation}
Note that the supremum must be over $v$ such that $v-u_l \in \K$,
otherwise $-\F(v) = -\infty$.  If $v-u_l\in \K$ consider $\mu$ such that
\begin{equation}
  -\intO \mu_{ij} \epsij(v) \dx + \intO f_i v_i \dx
  + \intG[1] t_i v_i \ds = 0 \quad\text{for all }v\in \V (v-u_l\in \V \text{??})
\end{equation}
If the above do not hold, there is a function $v$ such that it is
positive and $cv$ for $c\in\R$ can make the expression as
large as we wish, thus $-\F(v)$ is $\infty$.  Using Greens theorem
\begin{multline}
  \intO \dj \mu_{ij} v_i \dx - \intG[1] \nu_j \mu_{ij} v_i \ds
  - \intG[c] \nu_j \mu_{ij} v_i \ds  \\
  + \intO f_i v_i \dx
  + \intG[1] t_i v_i \ds = 0 \quad\text{for all } v\in \V 
\end{multline}
Using this it is straightforward to see that
\begin{equation}
  \label{eq:stressEquilibrium}
  \begin{split}
    -\dj \mu_{ij} &= f_i,\quad\text{in }\Omega \\
    \nu_j \mu_{ij} &= t_i,\quad\text{on }\Gamma_1 \\
    (\nu_j \mu_{ij})^\tau_i &= 0,\quad\text{on }\Gamma_c
  \end{split}
\end{equation}
for $i=1,\ldots,d$.  Whenever $\mu^\nu \le 0$ on $\Gamma_c$ holds it follows that
\begin{equation}
  \intG[c] \mu^\nu \;v^\nu \ds
  \le \intG[c] \mu^\nu \;l \ds
\end{equation}
Note that in general $v^\nu \in H^\half(\Gamma_c)$ and $\mu^\nu \in
\bigl(H^\half(\Gamma_c)\bigr)$, thus the above inequality can be written
\begin{equation}
  \dualp{\mu^\nu}{v^\nu}_{\Gamma_c} \le \dualp{\mu^\nu}{l}_{\Gamma_c}
\end{equation}

The conjugate of $\G$ is
\begin{equation}
  \G^*(\eta) = \sup_{\mu\in \W} \{ \intO \eta_{ij} \mu_{ij} \dx
  - \half \intO \mu_{ij} C_{ijkl} \mu_{kl} \dx \}
\end{equation}
Since the functional on the right hand side is differentiable the
supremum is attained whenever
\begin{equation}
  \label{eq:epsopt}
  \intO \eta_{ij} \mu_{ij} \dx =  \intO \mu_{ij} C_{ijkl} \mu_{kl} \dx
  \quad \text{for all $\mu$ in \W}
\end{equation}
i.e.
\begin{equation}
  \eta_{ij} = C_{ijkl} \mu_{kl}  \quad\text{or}\quad
  \mu_{kl} = C^{-1}_{ijkl} \eta_{ij}
\end{equation}
Consequently
\begin{equation}
  \G^*(\eta) = \half\intO \eta_{ij} C^{-1}_{ijkl} \eta_{kl} \dx
\end{equation}

The right hand side of~(\ref{eq:PrimalDual}) the becomes
\begin{equation}
  \label{eq:dual_mu}
  \sup_{\substack{
      \mu\in \W\\
      \mu^\nu \le 0 \text{ on }\Gamma_c\\
      (\ref{eq:stressEquilibrium})\text{ is satisfied}
    }}
    -\half\intO \mu_{ij} C^{-1}_{ijkl} \mu_{kl} \dx
  + \dualp{\mu^\nu}{l}_{\Gamma_c}
\end{equation}

Since $\G$ is differentiable the subdifferential of $\G$ at
$\epskl(u)$ is given by $C_{ijkl} \epskl(u)$ and since $\sigma\in
\partial \G(\strain(u))$
\begin{equation}
  \label{eq:stressStrainOptimal}
  \sigij(u) = C_{ijkl} \epskl(u)
\end{equation}
Note that by (\ref{eq:dual_mu}) the normal stress $\sigma^\nu$ on $\Gamma_c$ is negative.
 

The optimality condition $-\strain^*(\sigma) \in \partial \F(u)$
is, by definition,
\begin{equation}
  \F(v) - \F(u) \ge \dualp{-\epsilon^* \sigma}{v-u} \quad\text{for all $v\in \V$}
\end{equation}
If $v-u_l\in \K$
\begin{equation}
  \int_{\Omega}\!\! \sigij \epsij(v-u) \dx - F(v-u) \ge 0
\end{equation}
i.e.\
\begin{equation}
  A(u, v-u) \ge F(v-u) \quad\text{for all $v-u_l\in \K$}
\end{equation}

Since the relation (\ref{eq:stressStrainOptimal}) is satisfied for the optimal solution,
it makes sense to look for $\mu\in\W$ satisfying this relation.  Let $\lambda \in
\bigl(H^\half(\Gamma_c)\bigr)^*$ and consider $v \in \V$ satisfying
\begin{equation}
  \label{eq:varNormalStress}
  A(v_\lambda, v) = F(v) - \dualp{\lambda}{v^\nu}_{\Gamma_c}
\end{equation}
Here $\lambda = -\sigma^\nu(v_\lambda)$.  Using this the dual problem can be written
\begin{equation}
  \min_{\substack{
      \lambda\in ( H^\half(\Gamma_c) )^* \\
      \lambda \ge 0}
  } \half A(v_\lambda, v_\lambda) + \dualp{\lambda}{l}_{\Gamma_c}
\end{equation}
Note that if $\lambda$ minimization the above functional is known, the corresponding
$u_\lambda$ is the solution of the primal minimization problem.  Thus if the normal stress
$\sigma^\nu$ is known, the solution $u$ of the contact problem
satisfies~(\ref{eq:varNormalStress}).


\subsection{Alternative method}

Since \K\ is a convex cone with vertex at zero, it can be shown that $w\in \K$
satisfies
\begin{align}
  &A(w,\phi) \ge F(\phi) - A(u_l,\phi) \quad\forall{\phi\in \K} \\
  &A(u,u-u_l)  = F(u-u_l)
\end{align}
where $w=u-u_l$ and $\phi=v-u_l$.

Let $\V_0 = \set{v\in \V | v^\nu = 0 \text{ on }\Gamma_c}$. Clearly
\begin{equation}
  A(w,\phi) = F(\phi) - A(u_l, \phi) \quad\forall{\phi\in\V_0}
\end{equation}
Using Greens formula it follows that
\begin{align}
  \label{eq:elasteq}
  -\dj C_{ijkl}\epskl(u) &= f_i \quad\text{in $\Omega$ for $i=1,\ldots,d$} \\
  \nu_j C_{ijkl}\epskl(u) &= t_i \quad\text{on $\Gamma_1$ for $i=1,\ldots,d$} \\
  \sigma^\tau(u) &= 0 \quad\text{on $\Gamma_c$}
\end{align}
Multiplying the first equation with $\phi\in \K$ and using Greens formula result in
\begin{equation}
  A(u,\phi) = \intO f_i \phi_i\dx + \intG[1] t_i \phi_i \ds
  + \intGc \sigma^\nu \phi^\nu \ds
\end{equation}
Subtracting ... result in 
\begin{equation}
  \intGc \sigma^\nu \phi^\nu \ds \ge 0 \qforall \phi\in\K
\end{equation}
Since $\phi^\nu\le 0$ on $\Gamma_c$, $\sigma^\nu \le 0$ on $\Gamma_c$. Note that this can
also be written
\begin{equation}
  \intGc \sigma^\nu (v^\nu-l) \ds \ge 0 \qforall v-u_l\in\K
\end{equation}
Moreover
\begin{equation}
  \intGc \sigma^\nu (u^\nu - l) \ds = 0
\end{equation}
thus, on $\Gamma_c$ either $\sigma^\nu = 0$ or $u^\nu = l$.

Introduce the Lagrangian $\Lagrange:\V\times \Lambda$ by
\begin{equation}
  \Lagrange(v,\eta) = \half A(v,v) - F(v) + \dualp{\eta}{v^\nu-l}_{\Gamma_c}
\end{equation}
where $\Lambda=\set{\eta\in L^2(\Gamma_c) | \eta \ge 0 \text{ a.e. on }\Gamma}$.

The corresponding saddle point problem is: Find $(u,\lambda)$ such that
\begin{equation}
  \Lagrange(u,\eta) \le \Lagrange(u,\lambda) \le \Lagrange(v,\lambda)
  \qforall (v,\lambda) \in \V\times\Lambda
\end{equation}
Take $\lambda = -\sigma^\nu(u)$, where $u$ us the solution of the variational inequality.
Since $u^\nu -l \le 0$ on $\Gamma_c$ and $\lambda \ge 0$,
\begin{equation}
  \dualp{\eta}{u^\nu - l}_{\Gamma_c} \le 0 \qforall \eta\in \Lambda
\end{equation}
Therefore
\begin{multline}
  \Lagrange(u,\eta) = \half A(u,u) - F(u) + \dualp{\eta}{u^\nu - l}_{\Gamma_c}
  \le \half A(u,u) - F(u) \\
  = \half A(u,u) - F(u) + \dualp{\lambda}{u^\nu - l}_{\Gamma_c}
  = \Lagrange(u,\lambda)
\end{multline}
The right hand side inequality follows from the observation that if $\lambda=-\sigma^\nu$
is known the solution $u$ satisfies~(\ref{eq:varNormalStress}), and the corresponding
minimization problem...

The following Uzawa type algorithm can be used to solve the saddle point problem: Choose
$\lambda^0 \in \Lambda$ and the real number $\rho > 0$, then for $k=1,2,\ldots$ compute
$u^k\in\V$ satisfying
\begin{equation}
  \Lagrange(u^k, \lambda^{k-1}) \le \Lagrange(v, \lambda^{k-1})
  \qforall v \in V
\end{equation}
and $\lambda^k\in \Lambda$ is given by
\begin{equation}
  \lambda^k = P_\Lambda \bigl(\lambda^{k-1} - \rho (\nu\cdot u^k - l) \bigr)
\end{equation}
where $P_\Lambda$ is the projection from $L^2(\Gamma_c)$ into $\Lambda$.

The iteration converges if $0 < \rho < 2/\norm{\nu\cdot}$. Here the
norm is the norm of the normal trace operator.

The iteration is stopped when
\begin{equation}
  \nu\cdot u^k -l \le \epsilon_1 \quad\text{and}\quad
  \abs{\dualp{\lambda^k}{\nu\cdot u^k -l}_{\Gamma_c}} \le \epsilon_2 
\end{equation}
Another alternative is to use
\begin{equation}
  \abs{A(u^k, u^k-u_l) - F(u^k - u_l)} \le \epsilon
\end{equation}
Note that the minimization problem in the first step is equivalent to the variational
problem: Find $u^k\in \V$ satisfying
\begin{equation}
  A(u^k, v) = F(v) - \dualp{\lambda^{k-1}}{v^\nu}_{\Gamma_c}
  \qforall v\in \V
\end{equation}
\subsection{Alternative method, II}

Since \K\ is a convex cone with vertex at zero, it can be shown that $w\in \K$
satisfies
\begin{align}
  &A(w,\phi) \ge F(\phi) - A(u_l,\phi) \quad\forall{\phi\in \K} \\
  &A(u,u-u_l)  = F(u-u_l)
\end{align}
where $w=u-u_l$ and $\phi=v-u_l$.

Let $\V_0 = \set{v\in \V | [v]^\nu = 0 \text{ on }\Gamma_c}$. Clearly
\begin{equation}
  A(w,\phi) = F(\phi) - A(u_l, \phi) \quad\forall{\phi\in\V_0}
\end{equation}
Using Greens formula it follows that
\begin{align}
  \label{eq:elasteqFixedPoint}
  -\dj C_{ijkl}\epskl(u) &= f_i \quad\text{in $\Omega$ for $i=1,\ldots,d$} \\
  \nu_j C_{ijkl}\epskl(u) &= t_i \quad\text{on $\Gamma_1$ for $i=1,\ldots,d$} \\
  \sigma^\tau(u) &= 0 \quad\text{on $\Gamma_c$}
\end{align}
Multiplying the first equation with $\phi\in \K$ and using Greens formula result in
\begin{equation}
  A(u,\phi) = \intO f_i \phi_i\dx + \intG[1] t_i \phi_i \ds
  + \intGc \sigma^\nu [\phi]^\nu \ds
\end{equation}
Subtracting ... result in 
\begin{equation}
  \intGc \sigma^\nu [\phi]^\nu \ds \ge 0 \qforall \phi\in\K
\end{equation}
Since $\phi^\nu\le 0$ on $\Gamma_c$, $\sigma^\nu \le 0$ on $\Gamma_c$. Note that this can
also be written
\begin{equation}
  \intGc \sigma^\nu ([v]^\nu-l) \ds \ge 0 \qforall v-u_l\in\K
\end{equation}
Moreover
\begin{equation}
  \intGc \sigma^\nu ([u]^\nu - l) \ds = 0
\end{equation}
thus, on $\Gamma_c$ either $\sigma^\nu = 0$ or $u^\nu = l$.

Introduce the Lagrangian $\Lagrange:\V\times \Lambda$ by
\begin{equation}
  \Lagrange(v,\eta) = \half A(v,v) - F(v) + \dualp{\eta}{[v]^\nu-l}_{\Gamma_c}
\end{equation}
where $\Lambda=\set{\eta\in L^2(\Gamma_c) | \eta \ge 0 \text{ a.e. on }\Gamma}$.

The corresponding saddle point problem is: Find $(u,\lambda)$ such that
\begin{equation}
  \label{eq:SaddlePointII}  
  \Lagrange(u,\eta) \le \Lagrange(u,\lambda) \le \Lagrange(v,\lambda)
  \qforall (v,\lambda) \in \V\times\Lambda
\end{equation}
Take $\lambda = -\sigma^\nu(u)$, where $u$ us the solution of the variational inequality.
Since $[u]^\nu -l \le 0$ on $\Gamma_c$ and $\lambda \ge 0$,
\begin{equation}
  \dualp{\eta}{[u]^\nu - l}_{\Gamma_c} \le 0 \qforall \eta\in \Lambda
\end{equation}
Therefore
\begin{multline}
  \Lagrange(u,\eta) = \half A(u,u) - F(u) + \dualp{\eta}{[u]^\nu - l}_{\Gamma_c}
  \le \half A(u,u) - F(u) \\
  = \half A(u,u) - F(u) + \dualp{\lambda}{[u]^\nu - l}_{\Gamma_c}
  = \Lagrange(u,\lambda)
\end{multline}
The right hand side inequality follows from the observation that if $\lambda=-\sigma^\nu$
is known the solution $u$ satisfies~(\ref{eq:varNormalStress}), and the corresponding
minimization problem...

The following Uzawa type algorithm can be used to solve the saddle point problem: Choose
$\lambda^0 \in \Lambda$ and the real number $\rho > 0$, then for $k=1,2,\ldots$ compute
$u^k\in\V$ satisfying
\begin{equation}
  \Lagrange(u^k, \lambda^{k-1}) \le \Lagrange(v, \lambda^{k-1})
  \qforall v \in V
\end{equation}
and $\lambda^k\in \Lambda$ is given by
\begin{equation}
  \lambda^k = P_\Lambda \bigl(\lambda^{k-1} - \rho (\nu\cdot [u^k] - l) \bigr)
\end{equation}
where $P_\Lambda$ is the projection from $L^2(\Gamma_c)$ into $\Lambda$.

The iteration converges if $0 < \rho < 2/\norm{\nu\cdot}$. Here the
norm is the norm of the normal trace operator.

The iteration is stopped when
\begin{equation}
  \nu\cdot [u^k] -l \le \epsilon_1 \quad\text{and}\quad
  \abs{\dualp{\lambda^k}{\nu\cdot [u^k] -l}_{\Gamma_c}} \le \epsilon_2 
\end{equation}
Another alternative is to use
\begin{equation}
  \abs{A(u^k, u^k-u_l) - F(u^k - u_l)} \le \epsilon
\end{equation}
Note that the minimization problem in the first step is equivalent to the variational
problem: Find $u^k\in \V$ satisfying
\begin{equation}
  A(u^k, v) = F(v) - \dualp{\lambda^{k-1}}{[v]^\nu}_{\Gamma_c}
  \qforall v\in \V
\end{equation}


Let $\V_h$ be a finite element space, consisting of linear and/or bilinear polynomials if
$d=2$ and linear and/or trilinear polynomials if $d=3$.  The nodes on the contact boundary
is $a_i$, $i=1,\ldots,m$.  Let $\Lambda_h$ be a subspace of $\Lambda$ constructed using
piecewise linear functions for $d=2$ and linear and/or bilinear functions for $d=3$.  In
addition the nodal values are required to be nonnegative, i.e.~$\lambda_j\ge 0$ for
$j=1,\ldots,m$. Then $\lambda_h\subset \lambda$.

For $d=2$ and using the trapezoidal rule for integration we obtain
\begin{multline}
  \intGc \lambda([v]^\nu-l) \ds \\
  = \sum_{j=1}^{m-1} \frac{h_j}{2}\Bigl(\lambda(a_j)\bigl([v(a_j)]^\nu-l_j\bigr)
  + \lambda(a_{j+1})\bigl([v(a_{j+1})]^\nu-l_{j+1}\bigr)\Bigr) \\
  = \frac{1}{2} \Bigl( h_1\lambda(a_1)\bigl([v(a_1)]^\nu - l_1 \bigr)
  +  h_m\lambda(a_m)\bigl([v(a_m)]^\nu - l_m \bigr) \Bigr) \\
  + \sum_{j=2}^{m-1} h_j \lambda(a_j)\bigl([v(a_j)]^\nu-l_j\bigr)
  = \lambda\cdot(Bv - c)
\end{multline}
with appropriate definitions of the vectors $\lambda$ and $c$, and the matrix $B$.

In matrix notation the first step in the Uzawa algorithm becomes
\begin{equation}
  A u^k = F - B^T \lambda^{k-1}
\end{equation}
and the new approximation to the Lagrange multiplier takes the form
\begin{equation}
  \lambda^k = P_{\Lambda_h}\bigl(\lambda^{k-1} - \rho(B u^k -c)\bigr) 
\end{equation}
Here the projection operator set negative elements in the vector to zero, the positive
elements are not changed by the projection.

For a given $\lambda$ let $\ulambda\in \V$ satisfy
\begin{equation}
  A(\ulambda, v) = F(v) - \dualp{\lambda}{[v]^\nu}_{\Gamma_c}
  \qforall v\in \V
\end{equation}
Using this the left hand optimization problem in the Saddle point
problem~\ref{eq:SaddlePoint} is an optimization problem involving only $\lambda$:
\begin{equation}
\half A(\ulambda,\ulambda) - F(\ulambda) + \dualp{\lambda}{[\ulambda]^\nu - l}_{\Gamma_c}  
\end{equation}
Note that
\begin{align}
  A(\ulambdah, v) &= F(v) - \dualp{\lambda+h}{[v]^\nu}_{\Gamma_c} \\
  &= A(\ulambda, v) - \dualp{h}{[v]^\nu}_{\Gamma_c} \\
  &= A(\ulambda, v) + A(\ulh,v)
  \qforall v\in \V
\end{align}
Consequently
\begin{multline}
  A(\ulambdah, \ulambdah) = F(\ulambdah) - \dualp{\lambda+h}{[\ulambdah]^\nu}_{\Gamma_c} \\
  = A(\ulambda, \ulambdah) - \dualp{h}{[\ulambdah]^\nu - l}_{\Gamma_c} \\
  = A(\ulambda, \ulambdah) + A(\ulh,\ulambdah) - F(\ulambdah)
\end{multline}
and
\begin{align}
  A(\ulambdah, \ulambda) &= F(\ulambda) - \dualp{\lambda+h}{[\ulambda]^\nu - l}_{\Gamma_c} \\
  &= A(\ulambda, \ulambda) - \dualp{h}{[\ulambda]^\nu - l}_{\Gamma_c} \\
  &= A(\ulambda, \ulambda) + A(\ulh,\ulambda) - F(\ulambda)
\end{align}
and
\begin{equation}
  A(\ulambda, \ulambdah) = F(\ulambdah) - \dualp{\lambda}{[\ulambdah]^\nu - l}_{\Gamma_c} \\
\end{equation}
Now
\begin{multline*}
  \half A(\ulambdah,\ulambdah) - F(\ulambdah) + \dualp{\lambda+h}{[\ulambdah]^\nu -
    l}_{\Gamma_c}  \\
  = \half A(\ulambda,\ulambdah) + \half A(\ulh, \ulambdah) - A(\ulambda, \ulambdah) \\
  + \dualp{h}{[\ulambdah]^\nu-l}_{\Gamma_c}  \\
  = \half A(\ulambda, \ulambdah) - \half\dualp{h}{[\ulambdah]^\nu - l}_{\Gamma_c}
  -  A(\ulambda, \ulambdah) \\+ \dualp{h}{[\ulambdah]^\nu -
    l}_{\Gamma_c}  \\
  =   \half A(\ulambda,\ulambda) - F(\ulambda)  + \dualp{\lambda}{[\ulambda]^\nu -
    l}_{\Gamma_c} + \dualp{h}{[\ulambda]^\nu-l} \\ -\half A(\ulh,\ulh)
\end{multline*}





\section{A finite element method for the primal\\ variational formulation}
\label{sec:finite-elem-meth}

The finite element method discussed here is found in~\cite{haslinger04:3dContact}.

Assume that $\Omega$ is a polyhedral domain with a regular partition \Th\ into
tetrahedrons or hexahedrons.  The decomposition is consistent with the decomposition of
$\dOmega$ into $\Gamma_0$, $\Gamma_1$ and $\Gamma_c$.  The finite element subspace
$\Vh\subset\V$ is constructed using the usual linear or bilinear basis functions.  The
dimension of $\Vh$ is $d n$ where $n$ is the number of nodes from \Th\ in
$\overline{\Omega} \setminus \overline{\Gamma}_0$.

The nodes from \Th\ on $\overline{\Gamma}_c \setminus \overline{\Gamma}_0$ are denoted
$\{a_j\}_{j=1}^{m}$ and called \emph{contact nodes}.  At the node $a_j$ the displacement
vector is $v_j$, the outward normal vector of $\Gamma_c$ is $\nu^j$ and the unit tangent
vectors are $\tau_1^j$ and $\tau_2^j$.  The vectors $\nu^j$, $\tau_1^j$ and $\tau_2^j$
form an orthonormal basis for $\R^3$ for $j=1,\dots,m$.

The linear mappings $N$ and $T_k$ from $\Vh$ into $\R^m$ are defined by
\begin{equation}
  \label{eq:11}
  (Nv)_j = v^j\cdot\nu^j\quad\text{and}\quad (T_k v)_j = v^j\cdot \tau_k^j
\end{equation}
for $j=1,\dotsc,m$ and $k=1,2$.  Furthermore, the mapping $T$ is defined by $(Tv)_j =
\bigl( (T_1 v)_j, (T_2 v)_j \bigr)$.  The space of kinematically admissible finite element
functions is the space $\K_h$ where the normal component constraint is satisfied for all
the nodes at the contact boundary, i.e.
\begin{equation}
  \label{eq:4}
  \K_h = \{ v\in\Vh \colon (N v)_j \le d_j\}
\end{equation}
for all $j=1,\dotsc,m$.

Using the finite element method we obtain a stiffness matrix $K$ and a load vector $f$.
Representing the function $g$ using the sum of the Dirac distributions at the contact
nodes, the friction work function can be represented as
\begin{equation}
  \label{eq:3}
  \sum_{j=1}^m g_j \norm{(T v)_j}
\end{equation}
where $g_j \ge 0$ for $j=1,\dots,m$.  Using this the algebraic minimization problem is:
Find $u\in\Kh$ such that
\begin{equation}
  \label{eq:5}
  \J(u) \le \J(v) \qforall v\in\Kh
\end{equation}
where
\begin{equation}
  \label{eq:8}
  \J(v) = \half A(v,v) - F(v) + \sum_{j=1}^m g_j \norm{(T v)_j}
\end{equation}



\section{Numerical solution of the primal formulation}
\label{sec:numer-solut-prim}

In this section we mainly consider contact problem. This is a quadratic minimization
problem with inequality constraints.  The friction problem is more complicated since the
functional to be minimized is nondifferentiable.

The quadratic minimization problem~(\ref{eq:PrimalVar}) with the functional $J$ given
by~(\ref{eq:PotentialEnergyContact}) can, in principle, be solved using an algorithm for
the minimization of quadratic functional with inequality constraints.  In practice the
number of unknowns in the finite element discretization is large and the minimization
algorithms may not be sufficiently efficient.  Note that the constraints in the space $\Kh$
involves only a small number of the degrees of freedom of the system.  Consequently, it
should be possible to work only with the degrees of freedom on the contact boundary in the
minimization problem.  We now show one possible strategy for the reduction of the
primal minimization problem.

Let
\begin{equation}
  \label{eq:31}
  \Vh^0 = \{ v\in \V \colon v^\nu = 0\quad\text{on }\Gamma_c \}
\end{equation}
and 
\begin{equation}
  \label{eq:32}
  \W = \{ (Nv)_j,j=1,\dotsc,m \colon v\in \Vh\} \subset \Rm
\end{equation}
Moreover, define the constraint space
\begin{equation}
  \label{eq:38}
  \W_{-} = \{ w\in \W \colon w_j\le d_j \text{ for }j=1,\dotsc,m \}
\end{equation}
For $v\in\Vh$ define $v_P \in \Vh^0$ satisfying
\begin{equation}
  \label{eq:34}
  A(v_P, \phi) = A(v, \phi) \qforall \phi\in\Vh^0
\end{equation}
and let $v_H=v-v_P$.  Note that $v_H^\nu = v^\nu$ and
\begin{equation}
  \label{eq:39}
  A(v_H,\phi)=0 \qforall \phi\in\Vh^0
\end{equation}
Furthermore, note that $v_H$ is uniquely defined by $v^\nu\in \W$.

Using the above decomposition of functions in $\Vh$ in the energy functional we obtain
\begin{equation}
  \label{eq:35}
  \J(v) = \half A(v_H, v_H) - F(v_H)
  + \half A(v_P, v_P) - F(v_P)
\end{equation}
Define the functional
\begin{equation}
  \label{eq:46}
  \J_c(\gamma) = \half A(v_H(\gamma), v_H(\gamma)) - F(v_H(\gamma)) + C_P
\end{equation}
where $C_P = \half A(v_P, v_P) - F(v_P)$ is a constant term, and consider the minimization
problem: Find $\eta\in\W_{-}$ satisfying
\begin{equation}
  \label{eq:36}
  \J_c(\eta) \le \J_c(\gamma) \qforall \gamma\in\W_{-}
\end{equation}
This is a constrained minimization problem on the contact boundary, i.e. considerably
smaller than~(\ref{eq:5}).

In order to give a matrix formulation of the variational formulation above let $r$ be the
vector of nodal degrees of freedom of $v$, and partition the nodal vector $r$ into $r_1$
and $r_2$ where the first part corresponds to the degrees of freedom in $\Vh^0$ and the
second part corresponds to the degrees of freedom in $\W$.  The stiffness matrix $K$ is
partitioned accordingly
\begin{equation}
  \label{eq:37}
  K = 
  \begin{pmatrix}
    K_{11} & K_{12} \\ K_{12}^T & K_{22}
  \end{pmatrix}
\end{equation}
Now the matrix formulation of~(\ref{eq:34}) is
\begin{equation}
  \label{eq:33}
  K_{11} r_P = K_{11} r_1 + K_{12} r_2
\end{equation}
Furthermore, the matrix formulation of~(\ref{eq:39}) is
\begin{equation}
  \label{eq:40}
  K_{11} r_H + K_{12} r_2 = 0
\end{equation}
Using this in~(\ref{eq:35}) we obtain
\begin{equation*}
  \label{eq:41}
  A(v_H(\gamma), v_H(\gamma)) = (r_H^T, r_2^T)
  \begin{pmatrix}
    K_{11} & K_{12} \\ K_{12}^T & K_{22}
  \end{pmatrix}
  \begin{pmatrix}
    r_H \\ r_2
  \end{pmatrix}
  = r_2^T ( K_{22} - K_{12}^T K_{11}^{-1} K_{12} ) r_2^T
\end{equation*}
and
\begin{equation*}
  F(v_H(\gamma)) = R_1^T r_H + R_2^T r_2 = (R_2^T - R_1^T K_{11}^{-1} K_{12} ) r_2
\end{equation*}
Thus, (\ref{eq:36}) is a constrained quadratic minimization problem involving the Schur
complement system.


\section{A saddle point formulation of the algebraic minimization problem}
\label{sec:saddle-point-form}

In order to motivate the saddle point formulation, note that
\begin{equation}
  \label{eq:13}
  g_j \norm{(Tv)_j}
  = \max_{\substack{\mu^t_j\in\R^2 \\ \norm{\mu^\tau_j}^*\le g_j}} (\mu^\tau_j, (Tv)_j)
\end{equation}
Consequently
\begin{equation}
  \label{eq:14}
  \sum_{j=1}^m (\mu^\tau_j, (Tv)_j) \le \sum_{j=1}^m g_j \norm{(Tv)_j}
\end{equation}
for all $\mu^\tau_j\in\R^2$ with $\norm{\mu^\tau_j}^*\le g_j$ for $j=1,\dotsc,m$, with equality
for some $\mu^\tau_j$.

Introduce the Lagrangian $\L \colon \Vh\times \Lambda(g)$ by
\begin{equation}
  \label{eq:15}
  \L(v,\mu) = \half A(v,v) - F(v) + \sumjm \mu^\nu_j \bigl((Nv)_j-d_j\bigr)
  + \sumjm \mu^t_j \cdot (Tv)_j
\end{equation}
where $\mu=(\mu^\nu, \mu^\tau_1, \mu^\tau_2)$ is a vector of Lagrange multipliers.  The space
associated to the multipliers are
\begin{equation}
  \label{eq:16}
  \Lambda(g) = \Lambda_\nu \times \Lambda_\tau(g)
\end{equation}
where
\begin{equation}
  \label{eq:17}
  \Lambda_\nu = \{\mu^\nu \in\R^m \colon \mu^\nu_j \ge 0,\quad\text{for }j=1,\dotsc,m \}
\end{equation}
and
\begin{equation}
  \label{eq:18}
  \Lambda_\tau = \{\mu^\tau_j \colon \norm{\mu^t_j}^* \le g_j \quad\text{for }j=1,\dotsc,m\}
\end{equation}

A saddle point formulation of the constrained minimization problem~(\ref{eq:5}) is: Find
$(u,\lambda)\in \Vh\times\Lambda(g)$ such that
\begin{equation}
  \label{eq:SaddlePoint}
  \L(u, \mu) \le \L(u,\lambda) \le \L(v,\lambda) \qforall (v,\mu) \in \Vh\times \Lambda(g)
\end{equation}

The solution of the saddle point problem~(\ref{eq:SaddlePoint}) is equivalent to:  Find
$(u,\lambda)\in \Vh\times\Lambda(g)$ such that
\begin{gather}
  \label{eq:LinElast}
  A(u,v) = F(v) - \sumjm \lambda^\nu_j (Nv)_j - \sumjm \lambda^\tau_j \cdot (Tv)_j
  \qforall v\in\V \\
  \begin{split}
  \sumjm \bigl((Nu)_j - d_j\bigr)&(\lambda^\nu_j - \mu^\nu_j) \\ 
  &+ \sumjm (Tu)_j \cdot (\lambda^\tau_j - \mu^\tau_j) \ge 0 \qforall \mu \in \Lambda(g)    
  \end{split}
  \label{eq:LagrangeMult}
\end{gather}
Note that~(\ref{eq:LinElast}) correspond to a linear elasticity problem with prescribed
normal and tangential stress given by $\lambda^\nu$ and $\lambda^\tau$.  Also, using the
Euclidean inner product the inequality (\ref{eq:LagrangeMult}) can be written
\begin{equation}
  \label{eq:LagrangeMultInner}
  (N u - d, \lambda^\nu-\mu^\nu) + (T u, \lambda^\tau-\mu^\tau) \ge 0
  \qforall \mu \in \Lambda(g)
\end{equation}
Alternatively, we also have
\begin{equation}
  \label{eq:20}
  (
  \begin{pmatrix}
    N \\ T
  \end{pmatrix}
  u, \lambda - \mu) \ge (
  \begin{pmatrix}
    d \\ 0
  \end{pmatrix}, \lambda - \mu) \qforall \mu\in\Lambda(g)
\end{equation}

It is possible to eliminate $u$ from the saddle point system~(\ref{eq:LinElast})
and~(\ref{eq:LagrangeMult}), and obtain a problem in the Lagrange multipliers.  To do this
define for $\mu\in\Lambda(g)$, $v(\mu)\in\Vh$ satisfying
\begin{equation}
  \label{eq:43}
  A(v(\mu), \phi) = -(N\phi, \mu^\nu) - (T\phi, \mu^\tau)
  \qforall \phi\in\Vh
\end{equation}
In addition, let
\begin{equation}
  \label{eq:42}
  A(u_f, \phi) = F(\phi)\qforall \phi\in\Vh
\end{equation}
Then $u=u(\lambda)+u_f$.  Now, define
\begin{equation}
  \label{eq:45}
  C(\eta, \mu) = -(Nv(\eta), \mu^\nu) - (Tv(\eta), \mu^\tau)
\end{equation}
Note that using~(\ref{eq:43})
\begin{equation}
  \label{eq:47}
  C(\eta, \mu) = A(v(\mu), v(\eta))
\end{equation}
Hence the bilinear form $C$ is symmetric, it can also be shown that it is coercive and
bounded.

Using the definition of $C$ in~(\ref{eq:LagrangeMultInner}) it follows that
\begin{equation}
  \label{eq:48}
    C(\lambda, \mu-\lambda) \ge (N u_f-d, \mu^\nu-\lambda^\nu)
    + (T u_f, \mu^\tau-\lambda^\tau)
  \qforall \mu \in \Lambda(g)
\end{equation}
Since $C$ is symmetric it is well known that this variational inequality problem is
equivalent to the constrained minimization problem: Find $\lambda\in\Lambda(g)$ satisfying
\begin{equation}
  \label{eq:49}
  \J_c (\lambda) \le \J_c(\mu)\qforall \mu\in\Lambda(g)
\end{equation}
where
\begin{equation}
  \label{eq:50}
  \J_c(\mu) = \half C(\mu,\mu) - (N u_f-d, \mu^\nu) - (T u_f, \mu^\tau)
\end{equation}

The operator form of~(\ref{eq:LinElast}) is
\begin{equation}
  \label{eq:1}
    A u = F - N^* \lambda^\nu - T^* \lambda^\tau \\
\end{equation}
where $N^*$ and $T^*$ from $\R^m$ into $\V^*$ are the dual operators of $N$ and
$T$.  The corresponding matrix formulation is
\begin{equation}
  \label{eq:2}
  K u = f - N^T \lambda^\nu - T^T \lambda^\tau.
\end{equation}
Clearly
\begin{equation}
  \label{eq:51}
  u = u_f + u(\lambda) = K^{-1} F - K^{-1}(N^T \lambda^\nu - T^T \lambda^\tau)
\end{equation}
Using this in the definition of $C$
\begin{equation}
  \label{eq:52}
  C(\lambda, \mu) = (
  \begin{pmatrix}
    N \\ T
  \end{pmatrix}
  K^{-1}
  \begin{pmatrix}
    N^T & T^T
  \end{pmatrix}
  \lambda, \mu)
\end{equation}
Moreover,
\begin{equation}
  \label{eq:53}
    (\begin{pmatrix}
    N u_f - d  \\ T u_f
  \end{pmatrix}
  , \mu) = 
  (\begin{pmatrix}
   N K^{-1} f - d \\
   T K^{-1} f
  \end{pmatrix}
  , \mu)
\end{equation}
This is the matrix representation of the bilinear form $C$.  Introduce the matrix
\begin{equation}
  \label{eq:23}
  Q = 
  \begin{pmatrix}
    N \\ T
  \end{pmatrix}
  \Kinv
  \begin{pmatrix}
    N^T & T^T
  \end{pmatrix}
  =
  \begin{pmatrix}
    N \Kinv N^T & N \Kinv T_1^T & N \Kinv T_2^T \\
    T_1 \Kinv N^T & T_1 \Kinv T_1^T & T_1 \Kinv T_2^T \\
    T_2 \Kinv N^T & T_2 \Kinv T_1^T & T_2 \Kinv T_2^T
  \end{pmatrix}
\end{equation}
and the vector
\begin{equation}
  \label{eq:24}
  h = 
  \begin{pmatrix}
    N\Kinv f - d \\ T_1 \Kinv f \\ T_2 \Kinv f
  \end{pmatrix}
\end{equation}
We have the matrix formulation
\begin{equation}
  \label{eq:22}
  (Q\lambda, \mu-\lambda) \ge (h, \mu-\lambda)\qforall \mu\in\Lambda(g)
\end{equation}

Note that the matrix $Q$ is symmetric and positive definite, hence~(\ref{eq:22}) is
equivalent to the dual minimization problem: Find $\lambda\in\Lambda(g)$
such that
\begin{equation}
  \label{eq:25}
  D(\lambda) \le D(\mu) \qforall \mu\in\Lambda(g)
\end{equation}
where the functional $D(\mu)$ is
\begin{equation}
  \label{eq:26}
  D(\mu) = \half (Q\mu, \mu)-(h, \mu)
\end{equation}
This is a constrained quadratic optimization problem and the form of the constraints
depend on the norm $\norm{\cdot}$.

Note that the dual minimization problem determine approximations to the normal and
tangential stress $\lambda^\nu$ and $\lambda^\tau$ for a given normal contact stress
$g=-\beta\sigma^\nu(u(g))$.  Furthermore, recall the the normal contact stress satisfy
$g=\Phi(g)$, cf.\ equation~(\ref{eq:12}).  Once $\lambda^\nu$ and $\lambda^\tau$ is found,
the displacements are found from~(\ref{eq:2}).

Also note that the matrix $Q$ may be computed at the cost of $m$ multiplications with $N^T$
and $T^T$, $3m$ solves of linear systems with $K$ as a coefficient matrix and $m$
multiplications with $N$ and $T$.


\section{A saddle point formulation of the contact problem}
\label{sec:saddle-point-contact}

Here we consider the contact problem.  The Lagrangian $\L \colon \Vh\times \Lambda^\nu$ is
\begin{equation}
  \label{eq:LagrangianContact}
  \L(v,\mu) = \half A(v,v) - F(v) + \sumjm \mu^\nu_j \bigl((Nv)_j-d_j\bigr)
\end{equation}
A saddle point formulation of the constrained minimization problem~(\ref{eq:5}) is: Find
$(u,\lambda)\in \Vh\times\Lambda_\nu$ such that
\begin{equation}
  \label{eq:SaddlePointContact}
  \L(u, \mu) \le \L(u,\lambda) \le \L(v,\lambda) \qforall (v,\mu) \in \Vh\times \Lambda_\nu
\end{equation}

The solution of the saddle point problem~(\ref{eq:SaddlePoint}) is equivalent to:  Find
$(u,\lambda)\in \Vh\times\Lambda^\mu$ such that
\begin{gather}
  \label{eq:LinElastContact}
  A(u,v) = F(v) - \sumjm \lambda^\nu_j (Nv)_j
  \qforall v\in\V \\
  \begin{split}
  \sumjm \bigl((Nu)_j - d_j\bigr)&(\lambda^\nu_j - \mu^\nu_j) \ge 0
  \qforall \mu \in \Lambda_\nu    
  \end{split}
  \label{eq:LagrangeMultContact}
\end{gather}

The matrix form of~(\ref{eq:LinElastContact}) is
\begin{equation}
  \label{eq:MatrixFormContact}
  K u = f - N^T \lambda^\nu.
\end{equation}
Using the Euclidean inner product inequality (\ref{eq:LagrangeMult}) can be written
\begin{equation}
  \label{eq:LagrangeMultContactInner}
  (N u - d, \lambda^\nu-\mu^\nu) \ge 0
  \qforall \mu \in \Lambda_\nu
\end{equation}
Eliminating $u$ using~(\ref{eq:MatrixFormContact}) gives
\begin{equation}
  \label{eq:21}
  (N \Kinv N^T \lambda^\nu, \mu^\nu-\lambda^\nu)
  \ge (N \Kinv f - d, \mu^\nu-\lambda^\nu) \qforall \mu\in\Lambda_\nu
\end{equation}
Since $N$ has full rank the matrix $N\Kinv N^T$ is symmetric and positive definite.
Consequently the variational inequality is equivalent to the quadratic minimization
problem: Find $\lambda^\nu\in \Lambda_\nu$ such that
\begin{equation}
  \label{eq:27}
  D(\lambda) \le D(\mu)\qforall \mu \in \Lambda_\nu
\end{equation}
where
\begin{equation}
  \label{eq:28}
  D(\mu) = \half(N \Kinv N^T \mu, \mu) - (N \Kinv f -d, \mu) 
\end{equation}
Recall that the space $\Lambda_\nu$ is the set of $m$ vectors with nonnegative entries.
Hence the minimization problem is a quadratic minimization problem with one sided
inequality constraints.


% \section{Multibody contact}

% Let the domain $\Omega$ consist of two bodies, $\Omega^1$ and
% $\Omega^2$. A contact zone is identified as $\Gamma^1_c$ and
% $\Gamma^2_c$, possibly with an initial gap.  The unit normal $\nu$ of the
% contact zone is taken to be the normal pointing from $\Omega^1$ into
% $\Omega^2$.  The initial gap is $l=l(x)$, i.e. the distance from
% $\Gamma^1_c$ to $\Gamma^2_c$ along the unit normal $\nu$. 

% The contact condition is
% \begin{equation}
%   u^1\cdot\nu - u^2\cdot\nu \le l \quad\text{on $\Gamma_c$}
% \end{equation}
% where we have used $\Gamma_c$ to refer to the contact boundary.
% Alternative notation
% \begin{equation}
%   [u]^\nu \le l \quad\text{on $\Gamma_c$}
% \end{equation}

% Let
% \begin{equation}
%   \K = \set{v\in \V \colon [v]^\nu \le 0 \quad\text{on }\Gamma_c }
% \end{equation}
% and let $u_l$ be a function in $\V$ such that $[u_l]^\nu = l$.
% Furthermore let
% \begin{equation}
%   J(v) = \half A(v,v) - F(v)
% \end{equation}

% Find $u-u_l \in \K$ such that
% \begin{equation}
%   \label{eq:primalMin}
%   J(u) \le J(v) \quad\text{for all $v-u_l \in \K$}
% \end{equation}
% or, equivalently, find $u-u_l \in \K$ such that such that
% \begin{equation}
%   A(u, v-u) \ge F(v-u) \quad\text{for all $v-u_l \in \K$}
% \end{equation}

% In order to derive the dual problem corresponding
% to~\ref{eq:primalMin}, introduce the functionals $\F:\V \mapsto \R$
% and $\G:\W \mapsto \R$, where $\W=\set{\mu\in L^2(\Omega)^{2,2} : \mu_{ij}=\mu_{ji}}$, by
% \begin{equation}
%   \F(v) =
%   \begin{cases}
%     -F(v) & \text{if $v-d\in\K$} \\
%     \infty & \text{otherwise}
%   \end{cases}
% \end{equation}
% and
% \begin{equation}
%   \G(\mu) = \half \int_\Omega \mu_{ij} C_{ijkl} \mu_{kl} \dx
% \end{equation}




\bibliography{bib/trustPapers,bib/trustBooks,bib/bookrefs}
\bibliographystyle{amsplain}


\end{document}
